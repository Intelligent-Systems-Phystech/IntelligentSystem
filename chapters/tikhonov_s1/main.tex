\section{Introduction}

Работа посвящена аппроксимации квазипериодических временных рядов.
Примерами таких рядов являются показатели акселерометра и гироскопа во время повторяющейся физической активности, электрокардиограмма.
	
В этой работе предлагается модель аппроксимации временных рядов.
Для этого производится переход в пространство фазовых траекторий или траекторное пространстве .
Переход осуществляется методом задержек.
Метод задержек используется при анализе нестационарных временных рядов.
Например, в методе сингулярного спектрального анализа разложения на компоненты и прогноз основаны на траекторной матрице.
Она позволяет перейти от скалярного временного ряда к многомерному представлению.
Метод задержек так же получили широкое распространение в анализе нелинейных динамических систем.

Избыточная размерность траекторного пространства приводит к неустойчивости исследуемых моделей и избыточно сложному описанию временного ряда.
Для понижения размерности фазового пространства предлагается использовать  метод главных компонент.

В выбранном пространстве пониженной размерности фазовая траектория проецируется на $p$-мерную единичную сферу.
Полученную на поверхности сферы функцию предлагается представить в виде ряда разложенного по сферическим гармоникам.
Радиус восстанавливается линейной моделью по имеющимся значениям углов.

\section{References Review}
В работе ~\cite{Motrenko2015} рассматривается общий подход к анализу фазовых траекторий для квазипериодических временных рядов.
В работе ~\cite{Usmanova2020} исследуется фазовая траектория в сферических координатах и методы ее построения и анализа.
В работе ~\cite{Hajarian2015} сферические гармоники применяются для параметризации поверхности сферы.


\section{Main Part}

Пусть имеется квазипериодический временной ряд $\mathbf{s}=[s_1,...,s_N]^{\mathsf{T}}$
Временной ряд $\{s_t\}_{t=1}^N$\, назовем \emph{квазипериодическим} с периодом $T$, если для всех $t\in[0,+\infty)\,$ найдется $\delta$, такое что для любого $l \in \mathbb{N},$ выполнено
\[
   s_{t} \approx s_{t + lT + \delta}, \quad |\delta| \ll T.
\]
Требуется построить некоторую функцию $g(\cdot)$ аппроксимирующую квазипериодической временной ряд.

\textbf{Задачи:} $ t \mapsto \mathbf{x} \mapsto \mathbb{H}_{s}^{n} \xrightarrow{} \mathbb{H}_{x}^{p} \xrightarrow{} \mathbb{S}_z^{p} \hookrightarrow [0,2\pi)$
    
    $\mathbb{H}_{s}^{n}$~--- исходное фазовое пространство;
    
    $\mathbb{H}_{x}^{p}$~--- фазовое подпространство в декартовых координатах;
    
    $\mathbb{S}_{z}^{p}$~--- фазовое подпространство в сферических координатах.
    
    \medskip
    Требуется построить:
    \begin{enumerate}
        \item модель $g(\cdot)$, аппроксимирующую фазовую траекторию
    \[ g: \mathbb{R}^{q_1} \times \mathbf{A} \xrightarrow{} \mathbb{S}_{z}^{p},\]
    
        \item модель $f(\cdot)$, восстанавливаюую фазу сигнала
    \[ f: \mathbb{R}^{q_2} \times \mathbb{S}_{z}^{p} \xrightarrow{} [0,2\pi),\]
    
    где $q_1,q_2$ - число параметров модели, $\mathbf{A} = [0,\pi]\times[0,2\pi)\times \dots \times [0,2\pi)$.
    \end{enumerate}
Для векторизации одномерного временного ряда исмользуется вектор задержек.
\[\mathbf{S} = 
\begin{bmatrix} 
	s_{1} & \ldots & s_{n}\\
	s_{2} & \ldots & s_{n+1}\\
	\vdots& \ddots & \vdots\\
	s_{N-n+1}&\ldots &s_{N}\\
\end{bmatrix} =
	\begin{bmatrix} 
      	\mathbf{s}_{1}\\
      	\mathbf{s}_{2}\\
      	\vdots\\
      	\mathbf{s}_{N-n+1}\\
   \end{bmatrix},
   \quad
   \mathbf{s}_{i} \in \mathbb{H}_{s}^{n}
   \]
где $n$~--- ширина окна.

Построенное таким образом векторное пространтсво позволяет исследовать свойства рядов с учетом их нестационарности. В таком пространме точки для временного ряда с разными, но не сильно отличающимися частотами и амплитудами будут находится в некоторой окересности друг друга.
Однако Метод PCA для построения фазового подпространства меньшей размерности $p \ll n$.
\[\mathbf{X} = \mathbf{S W} =
    \begin{bmatrix}
        \mathbf{x}_1 \\
        \mathbf{x}_2  \\
        \dots \\
        \mathbf{x}_{N-n+1}
    \end{bmatrix},
    \quad
    \mathbf{x}_{i} \in \mathbb{H}_{p}^{x},\]
    где $\mathbf{W}$ --- матрица вращения.
Далее производится переход в сферические координаты в полученном подпространстве $\mathbb{H}_{x}^{p}$.
Строится отображение из декартовых координат в сферические $\mathbb{H}_{x}^{p} \xrightarrow{} \mathbb{S}_{z}^{p}$:
\[
    \phi: \mathbf{x} \xrightarrow{} \mathbf{z} = [r,\alpha_{p-1},\dots,\alpha_1],
    \quad
    \mathbf{a} = [\alpha_{p-1},\dots,\alpha_1]
\]

\begin{enumerate}
    \item Предлагается аппроксимировать фазовую траекторию на поверхности сферы с помощью сферических гармоник:
        \[f_{\text{sp}}(\mathbf{w}_{sp},\mathbf{a}) = \sum_{l_{p-1} = 0}^{N_{\text{approx}}}\sum_{l_{p-2} = 0}^{l_{p-1}}...\sum_{l_1 = -l_2}^{l_2}
        w_{l_{p-1},...,l_1} Y_{l_{p-1},...,l_1}(\mathbf{a})\],
    
    \item Базисные функции представимы в виде     \[Y_{l_{p-1},...,l_1}(\mathbf{a}) = 
        \left[\prod\limits_{k = 2}^{p-1}{_k}{\overline{P}}_{l_k}^{l_{p-1}}(\alpha_k)\right]
        \frac{1}{\sqrt{2\pi}}
        \exp{(i l_1 \alpha_1)}\],
        где $l_{p-1},...,l_1$ - индексы удовлетворяющие
        \[l_{p-1} \geq l_{p-2} \geq \dots \geq l_2 \geq|l_1|.\]
    
    \item Решается задача 
        \[\mathbf{\hat{w}}_{sp} = {argmin}_{\mathbf{w}_{sp}}
        \|f_{\text{real}}(\mathbf{a}(t)) - f_{\text{sp}}(\mathbf{w}_{sp},\mathbf{a}(t))\|^2\].
\end{enumerate}

\begin{itemize}
\item Регрессионная модель $f_r(\cdot)$, восстанавливает по переменным $\mathbf{a} \in \hat{A}$ радиус $r$.
\[
    f_r: \mathbf{a}
    \mapsto
    \hat{r}
\]
\[
\hat{r} = f_r(\mathbf{\hat{w}}_r,\alpha_{1},\dots, \alpha_{p-1}) = \mathbf{a}^T\,\mathbf{\hat{w}}_r
\]
\[
\mathbf{\hat{w}_r} = \arg\min_{\mathbf{w}_r}(r - \mathbf{a}^T\,\mathbf{w}_r)^2
\]

\item Комбинация моделей  $ g = f_{sp}(\cdot)\cdot f_r(\cdot)$ позволяет с помощью подбираемых параметров $\mathbf{\hat{w}_r}$ и $\mathbf{\hat{w}_{sp}}$ полностью параметризовать фазовою траекторию вместе с областью дисперсии.
\end{itemize}
\section{Questions To Discussion}

Предложенная композиция методов позволяет уменьшить количество параметровс нескольких тысяч, в случае с полноценным автоэнкодером, до нескольких сотен.
В частности: предложен метод построения и уменьшения фазового пространства. метод построении векторов задержек и разложения на главные компоненты, метод аппроксимации фазовой траектории в сферическихкоординатах произвольной размерности с использованием сферическихгармоник и метод восстановления радиуса в пространства углов.
\begin{enumerate}
    \item Как выбрать ширину окна?
    \item Как выглядит на графике фазовой траектории точка соответ-
ствующая первому скользящему окну?
    \item Какое преимущество у данной модели надо точечной оценкой
мат ожидания?
\end{enumerate}