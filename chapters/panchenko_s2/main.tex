\section{Введение}

Конформная геометрическая алгебра -- это алгебра, конструируемая над образом биективного отображения из исходного (базового) пространства $\mathbb{R}^{p, q}$ в пространство более высокой размерности $\mathbb{R}^{p+1, q+1}$. Многие преобразования базового пространства, такие как отражения, повороты, трансляции могут быть представлены в виде элементов конформной геометрической алгебры; также геометрические объекты, такие как точки, прямые, сферы, приобретают естественные представления как элементы конформной алгебры.

\section{Литература}

	\begin{itemize}
		\item ~\cite{Sommer2001} -- обзор свойств и особенностей конформной геометрической алгебры;
		\item ~\cite{Heidelberg2009} -- обзор областей применения конформной геометрической алгебры
		\item ~\cite{Valkenburg2011} -- применение конформной геометрической алгебре в задаче поиска оптимальной композиции поворота-трансляции
	\end{itemize}
	
\section{Вопросы на обсуждение}

	\begin{itemize}
    \item Какая сигнатура у пространства, в которое погружается $\mathbb{R}^3$ для построения CGA?
    \item Что представляет собой motor?
    \item Если оператор $\mathcal{L}$ задаёт квадратичную форму, описывающую суммарную меру схожести после применения преобразования поворота, какой ротор будет наилучшим образом это преобразование описывать?
\end{itemize}