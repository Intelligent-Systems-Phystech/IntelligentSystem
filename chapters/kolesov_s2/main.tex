 

\section{Introduction}
На сегодняшний день сущесвтует несколько модификаций гибридного метода Монте-Кало на основе Гамильтоновой динамики.  Как известно данные алгоритмы в отличии от семейства алгоритмов Метрополиса-Хастингса способны совершать большие шаги по пространству, однако они никак не учитывают геометрию пространства, на котором эти семплы распределены. Таким образом, возникает идея посмотреть на Гамильтонову динамику с тчоки зрения геометрии, то есть с точки зрения того, а на каком манифолде лежат семплы.

Таким образом, давайте рассмотрим Гамильтонову динамику с геометрией.

\section{References Review}
Работа, что рассматривает геометрию для Гамилтоновой динамики в слкчае когда гамильтониан можно представить как сумму потенциальной и кинетиеческой энергии\cite{RMHMC}, однако если Гамильтониан в таком виде представить не получается, то тогда мы можем обратиться к работе , что связана Бэкер-Хаусдорф-Камбелл формулой для вычисления гамильтониана , и в работе\cite{SMHMC}

\section{main part}
Для начала давайте вспомним откуда возникает Гамильтониан для этого давайте рассмотри совместную плотность на семплы и вспомогательные переменные
 $$ p(x,v) =  \frac{1}{(2\pi)^{\frac{n}{2}} \sqrt{det \Sigma}} exp(- E(x) + \frac{1}{2}v^{T}\Sigma^{-1} v)$$
 Учитывая независимость переменных мы можем семплировать из совместной плтности и удаляя скорость текущую , мы получаем семпл на данном шаге. Однако, встает важный вопрос , а как собственно быть с кинетической энергией , что записана сверху как нормальное многовариантное распределение с некой матрицей коварицаии. И если определить так, то можно запустить симплектический интергируемый метод, напрмер 
 \[
     \systeme*{v(t + \epsilon) = p(t) - \epsilon \frac{\partial H}{\partial q(t)},
       q(t + \epsilon) = q(t) - \epsilon \frac{\partial H}{\partial v(t +\epsilon)}}
\]
И запуская данный метод, что является детерминированным, мы можем получать семплы. И как было отмечено ранее , что мы с вами делаем большие прыжки по пространству за счет гамильтониана , что птаемся везед определить. Однако, мы пока никак не учитываем геометрию. И тут оказывается, что на самом деле матрица ковариаций в данном случае является неким аналогом матрицы Грамма в неортонормированном базисе, то есть данная матрица ковариаций скоростей - задает скалярное произведение в пространстве, а значит она задает некую геометрию пространсва. И из теории гладких манифолдов мы знаем, что такую матрицу можно использовать как метрический тензор и такжеполучется , что наши данные определенны на гладком манифолде Римановском.
\\[0.2 cm]
Другими словами нам бы хотелось адаптировать локальную геометрию, то есть меньше шаг там, где плотность растет быстро.Определим понятие статистического манифолда
$$ S = {\mathcal{L}( : |x) : x \in X} $$
Также определим векторное поле на манифолде как
$$ v = \sum v^{j} \partial_{j} $$
И опрелелим первое тангентное пространство
$$ v^{(1)}  = \sum v^{j} \frac{\partial L}{\partial x_{j}}$$
И тогда получается что можно смотреть на матрицу ковариаций, как на следующую матрицу Фишера
$$ g_{x}(u,v) = \mathbb{E}_{l(|x)}[u^{(1)}v^{(1)}] $$



\section{Questions To Discussion}
Список вопросв по данному семинару
\begin{enumerate}
    \item Что является метрическим тензором на Римановском манифолде в случае гамильтновой динамики?
    \item  Что есть такое диффеоморфизм?
    \item  Что улучшает ланный метод, какую метрику по семплам? 
\end{enumerate}

 

