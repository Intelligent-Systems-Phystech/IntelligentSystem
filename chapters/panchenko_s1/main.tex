\section{Введение}

Геометрическая алгебра -- обобщение векторной линейной алгебры -- оперирует скалярами, векторами и линейными подпространствами как равноправными элементами алгебры, подчиняющимися операциям сложения и произведения. Геометрическая алгебра является полезным инструментом для описания геометрических объектов и преобразований над ними. В данной главе рассмотрены основы геометрической алгебры -- система образующих её аксиом и свойства её базовых операций.

\section{Литература}

	\begin{itemize}
		\item ~\cite{Hestenes1985} --  поверхностный обзор свойств и приемуществ геометрической алгебры;
		\item ~\cite{Hitzer2013} -- обзор областей применения геометрической алгебры;
		\item ~\cite{Chisolm} -- подробное изложение построения и анализа геометрической алгебры.
	\end{itemize}
	
\section{Вопросы на обсуждение}

	\begin{itemize}
	\item Верно ли, что геометрическая алгебра является, в частности, линейной алгеброй? Если да, то какова размерность описываемого ей пространства?
	\item Чему равно геометрическое произведение в случае коллинеарных и в случае ортогональных векторов?
	\item Что такое $r$-вектор в геометрической алгебре?
	\item Как записывается действие оператора отражения вектора вдоль оси в геометрической алгебре?
\end{itemize}